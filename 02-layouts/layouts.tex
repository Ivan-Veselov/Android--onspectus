\newcommand{\View}{
  \href{https://developer.android.com/reference/android/view/View.html}{View}
}

\newcommand{\ViewGroup}{
  \href{https://developer.android.com/reference/android/view/ViewGroup.html}{ViewGroup}
}

\Section{Макеты (Layouts)}

\Subsection{Макет}

Макет (англ. Layout) определяет то, как будет выглядеть структура UI. Создать
layout можно двумя способами:
\begin{itemize}
  \item \textbf{Описать элементы UI в XML-файле.} Андроид предоставляет
  прямолинейный XML синтаксис, который соответствует классам и подклассам
  \View.
  
  \item \textbf{Создать элементы layout во время выполнения.} Приложение может
  создавать объекты \View и \ViewGroup (и изменять их свойства) программно. 
\end{itemize}

Фреймворк Андроида позволяет гибко использовать только один или сразу оба из
этих способов, для создания и управления UI приложения.
Преимуществом создания UI в XML является тот факт, что внешний вид приложения
отделяется от кода, который управляет логикой.

\Subsection{Стандартные макеты}

Каждый подкласс класса \ViewGroup предоставляет уникальный способ для
отображения \View, которые он содержит внутри.

\textbf{Замечание:} Хотя и можно помещть один layout в другой, чтобы добиться
желаемого внешнего вида, не стоит этим злоупотреблять. Следует стараться
сделать иерархию как можно меньше. Layout рисуется быстрее, если у него мало
вложенных layout (широкая иерархия лучше, чем глубокая).

Вот некоторые наиболее часто используемые виды layout, которые встроены в
платформу Android:
\begin{itemize}
  \item \textbf{Linear Layout}. Layout, который распологает свои дочернии \View
  в один горизонтальный или вертикальный ряд. Он создает scrollbar (прокрутку),
  если размер содержимого превосходит размеры экрана.
  
  \item \textbf{Relative Layout}. Позволяет определить положение дочерних
  объектов относительно друг друга (например, расположить объект A слева от
  объекта B) или родителей (например, выровнить по верхней части родителя).
  
  \item \textbf{Web View}. Отображает web-страницы.
\end{itemize}

\begin{figure}[!htb]
\minipage{0.32\textwidth}
  \includegraphics[width=\linewidth]{02-layouts/linearlayout.png}
  \caption{Linear Layout}
\endminipage\hfill
\minipage{0.32\textwidth}
  \includegraphics[width=\linewidth]{02-layouts/relativelayout.png}
  \caption{Relative Layout}
\endminipage\hfill
\minipage{0.32\textwidth}
  \includegraphics[width=\linewidth]{02-layouts/webview.png}
  \caption{Web View}
\endminipage
\end{figure}

