\Section{Всплывающие сообщения (Toasts)}

\Subsection{Вывод всплывающего сообщения}

Приложение может показывать всплывающие сообщения с помощью класса Toast. У него есть статический метод makeText(), который создает View-элемент Toast. В качестве параметров он принимает:

\begin{itemize}

\item{\bf context} -- это нам пока не надо, просто напишем здесь $this$.

\item{\bf text} -- это текст, который будет показан на экране.

\item{\bf duration} -- продолжительность показа (Toast.LENGTH\_LONG - длинная, Toast.LENGTH\_SHORT - короткая).

\end{itemize}

Toast создан, и, чтобы он отобразился на экране, вызывается метод show(). 

\javacode{05-toasts/toast1.java}{}

Запускаем, смотрим и радуемся.

\image{05-toasts/Toast.png}{}


