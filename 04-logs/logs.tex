\Section{Логи}

\Subsection{Логи}

При тестировании приложения может быть полезно посмотреть логи работы. Они отображаются в окне logcat. Чтобы его найти, нужно тыкнуть кнопку Monitor, которая находится внизу слева.

\Subsection{Уровни логов}

У логов бывают разные уровни: Verbose, Debug, Info, Warn, Error, Assert. Как видно из названий, чем дальше, тем более специфичные вещи туда попадают. В окне logcat можно выбрать, какие именно уровни нам интересны. Тогда будут показаны сообщения, уровни которых не ниже того, что мы указали. 

\Subsection{Просмотр логов}

Вот мы запустили программу и хотим посмотреть логи. Для этого смотрим в окошко logcat и видим там следующую картину. 

\image{04-logs/logV.jpg}{тут показаны вообще все логи, которые есть}

Можем поменять уровень просматриваемых логов.

\image{04-logs/logI.jpg}{а тут только логи уровня Info и выше}

Также рядом расположена строка поиска по логам.

Посмотрим теперь на строки, которые нам выводятся. 
Каждая запись начинается с \textit{L/t:}, где \textit{L} обозначает уровень лога, а \textit{t} -- тег. Тег -- это что-то типа метки, чтобы легче было найти нужное сообщение.

\Subsection{Запись в логи}

Мы можем сами создавать логи. Делается это с помощью класса Log и его методов Log.v() Log.d() Log.i() Log.w() и Log.e(). Названия методов соответствуют уровню логов, которые они запишут. Есть еще метод Log.wtf(), который расшифровывают как "What a Terrible Failure!" и который записывает логи с уровнем Assert. 

У каждого такого метода два параметра -- тег и текст сообщения. Напишем по логу каждого уровня и посмотрим, что будет.

\javacode{04-logs/logs1.java}

Жмем Alt+Enter, и все компилируется.

Посмотрим, как работает.

\image{04-logs/Log3.jpg}{}

Собственно, все работает так, как мы и хотели, разве что Log.wtf() думает, что он не Assert, а Error. Ну бывает. 




